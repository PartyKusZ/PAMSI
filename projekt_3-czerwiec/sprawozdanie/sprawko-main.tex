\documentclass[12pt]{article}
\usepackage[utf8]{inputenc}
\usepackage[T1]{fontenc}
\usepackage[polish]{babel}
\usepackage{geometry}
\usepackage{tabularx}
\usepackage[table,xcdraw]{xcolor}
\usepackage{color}
\usepackage{subfig}
\usepackage{sidecap}
\usepackage{wrapfig}
\usepackage{float}
\usepackage{enumerate}
\usepackage{graphicx}
\usepackage{multirow}
\setlength{\parindent}{0pt}
\usepackage{hyperref}
\usepackage{titlesec}
\titlelabel{\thetitle.\quad}
\usepackage{amsmath}
\usepackage{anyfontsize}
\usepackage{indentfirst}
\usepackage{listings}
\usepackage{multicol}
\usepackage{pgfplots}
\usepackage{fancyhdr}

\definecolor{clr-background}{RGB}{255,255,255}
\definecolor{clr-text}{RGB}{0,0,0}
\definecolor{clr-string}{RGB}{163,21,21}
\definecolor{clr-namespace}{RGB}{0,0,0}
\definecolor{clr-preprocessor}{RGB}{128,128,128}
\definecolor{clr-keyword}{RGB}{0,0,255}
\definecolor{clr-type}{RGB}{59, 112, 230}
\definecolor{clr-variable}{RGB}{0,0,0}
\definecolor{clr-constant}{RGB}{111,0,138} % macro color
\definecolor{clr-comment}{RGB}{0,128,0}
\definecolor{mycolor}{rgb}{0.8,0.8,0.8}
\lstset{
  xleftmargin=20pt,
  xrightmargin=0pt,
  framexleftmargin=20pt,
  framexrightmargin=0pt,
  framexbottommargin=2pt,
  columns=flexible,
  keepspaces=true,
  showstringspaces=false,
  backgroundcolor=\color{clr-background},
  basicstyle=\color{clr-text}, % any text
  stringstyle=\color{clr-string},
  identifierstyle=\color{clr-variable}, % just about anything that isn't a directive, comment, string or known type
  commentstyle=\color{clr-comment},
  keywordstyle=\color{clr-type},
  tabsize=4,
  aboveskip=1em,
  belowskip=0em,
  frame=b,
  rulecolor=\color{mycolor},
  numbers=left,
  numbersep=10pt,
  numberstyle={\fontsize{9pt}{11pt}\selectfont\color{gray}},
}




\newgeometry{tmargin=1.8cm,bmargin=1.8cm,lmargin =1.8cm,rmargin=1.8cm}
\pagestyle{fancy}
\fancyhf{}
\rhead{\textit{Jakub Kusz}}
\lhead{\textit{ MiniMax z alfa-beta cięciami }}
\cfoot{ \thepage}

\begin{document}
    \begin{titlepage}
\begin{figure}
	\centering
	\includegraphics[width=18cm]{//home/kubus/Obrazy/logo-Pwr.png}
	
	\label{fig:pwr}
\end{figure}
	\begin{center}
		\huge Wydział Elektroniki, Fotoniki i Mikrosystemów \\ 
		\vspace{40pt}
		\huge PAMSI  \\
	\end{center}
	\vspace{60pt}
	\hrule
	\vspace{1pt}
	\hrule
	\begin{center}
		{\fontsize{40}{50}\selectfont Sprawozdanie nr 3\\ }
		\vspace{10pt}
		{\fontsize{25}{25}\selectfont Implementacja algorytmu MiniMax z alfa-beta cięciami w grze kółko i krzyżyk }
	\end{center}
	\hrule
	\vspace{1pt}
	\hrule
	\begin{flushright}
		\vspace{65pt}
		\textit{\Large Prowadzący:}\\
		
		\Large dr hab. inż. Andrzej Rusiecki\\
		\vspace{10pt}
		\textit{\Large Wykonał:}\\
		
		\Large Jakub Kusz \\
	
	\end{flushright}
	\vspace{90pt}
	\begin{center}
		\large Wrocław, \today r.
	\end{center}
\end{titlepage}



    \tableofcontents
    \newpage
    \section{Cel ćwiczenia}
        Celem ćwiczenia jest zaimplementowanie algorytmu MiniMax z alfa-beta cięciami w grze kółko i krzyżyk.
        Użytkownikowi dano możliwość wyboru poziomu trudności (głębokość algorytmu) i wielkości planszy.
    \section{Wstęp}
        \subsection{MiniMax}
            MinimMax jest algorytmem służącym do  minimalizowania maksymalnych możliwych strat. Alternatywnie można je traktować 
            jako maksymalizację minimalnego zysku. Wywodzi się to z teorii gry o sumie zerowej, obejmujących oba przypadki, 
            zarówno ten, gdzie gracze wykonują ruchy naprzemiennie, jak i ten, gdzie wykonują ruchy jednocześnie. 
            Zostało to również rozszerzone na bardziej skomplikowane gry i ogólne podejmowanie decyzji w obecności niepewności. 
            \subsubsection{Zasada działania}
                Najprostszym sposobem przedstawiania zasady działania algorytmu MiniMax jest ukazanie jego pracy na przykładzie.
                Na rysunku \ref{fig: drzewo}, pokazujemy wyniki uzyskane przez gracza x w każdym momencie gry. W bazie, na pierwszym poziomie, decyzję podejmuje przeciwnik. Z tego powodu podane są scenariusze, w których gracz może stracić -10 lub wygrać 5.
                Na drugim poziomie zależy to od gracza x, więc zmaksymalizuje on swój zysk. Pomiędzy stratą 10 a wygraną 1 wygrasz 1. Podobnie, jeśli wygrasz 5 lub 7, wygrasz 7.
                Potem znowu kolej na przeciwnika, więc zostaną podane scenariusze, w których gracz x ma najgorszy wynik, -3 i 4, w zależności od przypadku. Wreszcie, pomiędzy przegraną 3 a wygraną 4, gracz x podejmie decyzję, która pozwoli temu drugiemu.

                \begin{figure}[H]
                    \centering
                    \includegraphics[width = 15cm]{drzewo.jpeg}
                    \caption{Drzewo przeszukiwań}
                    \label{fig: drzewo}
                    
                \end{figure}

            \subsection{Alfa-beta cięcia}
            Korzyść płynąca z algorytmu alfa-beta leży w fakcie, że niektóre gałęzie drzewa przeszukiwania mogą zostać odcięte. Czas przeszukiwania ograniczony zostaje do przeszukania najbardziej obiecujących poddrzew, w związku z czym możemy zejść głębiej w tym samym czasie. Tak samo jak klasyczny min-max, algorytm należy do algorytmów wykorzystujących metody podziału i ograniczeń. Współczynnik rozgałęzienia jest dwukrotnie mniejszy niż w klasycznym  MiniMax. Algorytm staje się wydajniejszy, gdy węzły rozwiązywane są układane w porządku optymalnym lub jemu bliskim.
            
    \section{Implementacja}
        
        Z powodu ograniczeń związanych ze sprzętem nie jesteśmy w stanie rozwinąć całego drzewa od początku do końca, wiec algorytm MiniMax działa do 
        określonej głębokości, na której  następuje ocena pozycji za pomocą przeznaczonej do tego funkcji.Implementacja całej gry w języku C++ została umieszczona \href{https://github.com/PartyKusZ/PAMSI/tree/main/projekt_3-czerwiec}{\textbf{tutaj}}.
        \subsection{Ocena pozycji}
            Do oceny pozycji po osiągnięciu danej głębokości została napisana metoda  evaluate\_position(), która ocenia 
            pozycje na podstawie ilości kółek/krzyżyków z rzędu.
            \begin{lstlisting}[language=C++, caption=evaluate\_position()]
    void t_game :: evaluate_position(){
        int tmp;
        for(int i = 0; i < number_of_fields; ++i){
            tmp = 0;
            for(int j = 0; j < number_of_fields; ++j){
                if(gameborad_table[i][j] == 'o'){
                    tmp++;
                }else if(gameborad_table[i][j] == 'x'){
                    tmp = 0;
                    break;
                }
            }
            if(tmp){
                position_rating += pow(10,tmp);
            }
        }
    
        for(int i = 0; i < number_of_fields; ++i){
            tmp = 0;
            for(int j = 0; j < number_of_fields; ++j){
                if(gameborad_table[j][i] == 'o'){
                    tmp++;
                }else if(gameborad_table[j][i] == 'x'){
                    tmp = 0;
                    break;
                }
            }
            if(tmp){
                position_rating += pow(10,tmp);
            }
        }
        tmp = 0;
        for(int j = 0; j < number_of_fields; ++j){
            tmp = 0;
                if(gameborad_table[j][j] == 'o'){
                    tmp++;
                }else if(gameborad_table[j][j] == 'x'){
                    tmp = 0;
                    break;
                }
                if(tmp){
                    position_rating += pow(10,tmp);
                }
            }
            
        tmp = 0;
        for(int i = 0, j = number_of_fields - 1; i < number_of_fields; ++i, --j){
            tmp = 0;
                if(gameborad_table[i][j] == 'o'){
                    tmp++;
                }else if(gameborad_table[i][j] == 'x'){
                    tmp = 0;
                    break;
                }
                if(tmp){
                    position_rating += pow(10,tmp);
                }
            }
            
       
    }
                
            \end{lstlisting}
                
    

        \subsection{MiniMax z alfa-beta cięciami}
            Metoda znajdująca najlepsze zagranie:

            \begin{lstlisting}[language=C++, caption=minimax\_alpha\_beta()]
    int t_game :: minimax_alpha_beta(who_start current_player,  int depth, long int a, long int b){
    this->check_win();
    if(winner != who_start::draw){
        if(current_player == who_start :: ai){
            return INT32_MAX;
        }else{
            return INT32_MIN;
        }
    }
    if(this->is_finish() || depth == 0){
        if(current_player == who_start :: ai){
             this->evaluate_position();
             return position_rating;
        }else{
             this->evaluate_position();
             return -position_rating;
        }
    }
   long int best_score;
    if(current_player == who_start :: human){
        current_player = who_start :: ai;
        best_score = INT64_MIN;
    }else{
        current_player = who_start :: human;
        best_score = INT64_MAX;
    }
    int tmp;
    for(int i = 0; i < number_of_fields; ++i){
        for(int j = 0; j < number_of_fields; ++j){
            if(gameborad_table[i][j] == '_'){
                if(current_player == who_start :: ai){
                    gameborad_table[i][j] = 'o';
                    tmp = this->minimax_alpha_beta(current_player, depth-1, a, b);
                    if(best_score < tmp){
                        best_score = tmp;
                    }
                    if(a < best_score){
                        a = best_score;
                    }
                    gameborad_table[i][j] = '_';
                    if(a >= b){
                        return best_score;
                    }
                }else{
                    gameborad_table[i][j] = 'x';
                    tmp = this->minimax_alpha_beta(current_player, depth-1, a, b);
                    if(best_score > tmp){
                        best_score = tmp;
                    }
                    if(a > best_score){
                        a = best_score;
                    }
                    gameborad_table[i][j] = '_';
                    if(a >= b){
                        return best_score;
                    }
                }
            }
        } 
    }
    return best_score;
}
            \end{lstlisting}
    \subsection{Ustalenie współrzędnych najlepszego posunięcia}
    Do wyboru współrzędnych dla najlepszego posunięcia została napisana poniższa metoda:
    \begin{lstlisting}[language=C++, caption=best\_ai\_move()]
    void t_game ::  best_ai_move(int depth){

    long int best_score = INT64_MIN;
    int tmp;
    int set_i;
    int set_j;
    for(int i = 0; i < number_of_fields; ++i){
        for(int j = 0; j < number_of_fields; ++j){
            if(gameborad_table[i][j] == '_'){
                gameborad_table[i][j] = 'o';
                tmp = minimax_alpha_beta(who_start :: ai, depth, INT64_MIN, INT64_MAX);

                gameborad_table[i][j] = '_';
                if(tmp > best_score){
                    best_score = tmp;
                    set_i = i;
                    set_j = j;
                }
            }
        }
    }
    if(set_i < number_of_fields && set_j < number_of_fields)
        gameborad_table[set_i][set_j] = 'o';
}

\end{lstlisting}
\section{Testy}
    Gra została przetestowana pod względem poprawności działania i wydajności. 
        \subsection{Poprawność działania}
            Po rozegraniu wielu gier stwierdzono, że algorytm działa poprawnie, stara się wyszukiwać najlepsze ruchy,
            blokuje możliwości wygrania przez człowieka, najlepszą możliwością jest remis.
            Z obserwacji wynika, że już przy głębokości 1 nie człowiek nie jest w stanie wygrać z komputerem.
        \subsection{Wydajność}

        Aby gra przebiegała sprawie, dla danych rozmiarów planszy, zostały arbitralnie przydzielone maksymalne głębokości algorytmu:
        \begin{table}[H]
            \centering
            \caption{Zawierajaca czasy obliczania pierwszego ruchuchu algorytmu MiniMax dla danej wielkości planszy}
            \begin{tabular}{|c|c|c|}
            \hline
            Wielkość planszy & Głebokość maksymalna                                                                                           & Czas oczekiwania na pierwszy ruch {[}ms{]} \\ \hline
            3                & \begin{tabular}[c]{@{}c@{}}dowolna, algorytm jest w stanie \\ szybko rozwinąć grę do samego końca\end{tabular} & 90                                         \\ \hline
            4                & 5                                                                                                              & 1523                                       \\ \hline
            5                & 4                                                                                                              & 1816                                       \\ \hline
            6                & 3                                                                                                              & 480                                        \\ \hline
            7                & 3                                                                                                              & 1891                                       \\ \hline
            8                & 2                                                                                                              & 114                                        \\ \hline
            9                & 2                                                                                                              & 271                                        \\ \hline
            10               & 2                                                                                                              & 589                                        \\ \hline
            \end{tabular}
            \end{table}
           Zastosowanie większych głębokości dla danych rozmiarów planszy wiązało się z bardzo długim oczekiwaniem na ruch komputera w początkowej fazie gry. Wraz z każdym ruchem czas oczekiwania na odpowiedź komputera zmniejsza się. 
            \section{Wnioski}
            \begin{itemize}
                \item Algorytm MiniMax wraz z alfa-beta cięciami doskonale nadaje się do symulacji gracze w grze o sumie zerowej;
                \item Dzięki zastosowaniu alfa-beta cięć algorytm znacząco skraca czas działania, dzięki odcinaniu gałęzi niemających znaczenia dla rozwoju gry,
                \item Algorytm nawet na głębokości 1 jest w stanie skutecznie powstrzymać zwycięstwo człowieka, pozwalając maksymalnie na remis.
            \end{itemize}
\end{document}
