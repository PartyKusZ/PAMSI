\documentclass[12pt]{article}
\usepackage[utf8]{inputenc}
\usepackage[T1]{fontenc}
\usepackage[polish]{babel}
\usepackage{geometry}
\usepackage{tabularx}
\usepackage[table,xcdraw]{xcolor}
\usepackage{color}
\usepackage{subfig}
\usepackage{sidecap}
\usepackage{wrapfig}
\usepackage{float}
\usepackage{enumerate}
\usepackage{graphicx}
\usepackage{multirow}
\setlength{\parindent}{0pt}
\usepackage{hyperref}
\usepackage{titlesec}
\titlelabel{\thetitle.\quad}
\usepackage{amsmath}
\usepackage{anyfontsize}
\usepackage{indentfirst}
\usepackage{listings}
\usepackage{multicol}
\usepackage{pgfplots}
\usepackage{fancyhdr}



\definecolor{clr-background}{RGB}{255,255,255}
\definecolor{clr-text}{RGB}{0,0,0}
\definecolor{clr-string}{RGB}{163,21,21}
\definecolor{clr-namespace}{RGB}{0,0,0}
\definecolor{clr-preprocessor}{RGB}{128,128,128}
\definecolor{clr-keyword}{RGB}{0,0,255}
\definecolor{clr-type}{RGB}{59, 112, 230}
\definecolor{clr-variable}{RGB}{0,0,0}
\definecolor{clr-constant}{RGB}{111,0,138} % macro color
\definecolor{clr-comment}{RGB}{0,128,0}
\definecolor{mycolor}{rgb}{0.8,0.8,0.8}
\lstset{
  xleftmargin=20pt,
  xrightmargin=0pt,
  framexleftmargin=20pt,
  framexrightmargin=0pt,
  framexbottommargin=2pt,
  columns=flexible,
  keepspaces=true,
  showstringspaces=false,
  backgroundcolor=\color{clr-background},
  basicstyle=\color{clr-text}, % any text
  stringstyle=\color{clr-string},
  identifierstyle=\color{clr-variable}, % just about anything that isn't a directive, comment, string or known type
  commentstyle=\color{clr-comment},
  keywordstyle=\color{clr-type},
  tabsize=4,
  aboveskip=1em,
  belowskip=0em,
  frame=b,
  rulecolor=\color{mycolor},
  numbers=left,
  numbersep=10pt,
  numberstyle={\fontsize{9pt}{11pt}\selectfont\color{gray}},
}


\pgfplotsset{ticks=none}
\newgeometry{tmargin=1.8cm,bmargin=1.8cm,lmargin =1.8cm,rmargin=1.8cm}
\pagestyle{fancy}
\fancyhf{}
\rhead{Jakub Kusz}
\lhead{PAMSI}
\cfoot{ \thepage}

\begin{document}
    
\begin{titlepage}
\begin{figure}
	\centering
	\includegraphics[width=18cm]{//home/kubus/Obrazy/logo-Pwr.png}
	
	\label{fig:pwr}
\end{figure}
	\begin{center}
		\huge Wydział Elektroniki, Fotoniki i Mikrosystemów \\ 
		\vspace{40pt}
		\huge PAMSI  \\
	\end{center}
	\vspace{60pt}
	\hrule
	\vspace{1pt}
	\hrule
	\begin{center}
		{\fontsize{40}{50}\selectfont Sprawozdanie nr 3\\ }
		\vspace{10pt}
		{\fontsize{25}{25}\selectfont Implementacja algorytmu MiniMax z alfa-beta cięciami w grze kółko i krzyżyk }
	\end{center}
	\hrule
	\vspace{1pt}
	\hrule
	\begin{flushright}
		\vspace{65pt}
		\textit{\Large Prowadzący:}\\
		
		\Large dr hab. inż. Andrzej Rusiecki\\
		\vspace{10pt}
		\textit{\Large Wykonał:}\\
		
		\Large Jakub Kusz \\
	
	\end{flushright}
	\vspace{90pt}
	\begin{center}
		\large Wrocław, \today r.
	\end{center}
\end{titlepage}



\section{Zadanie}\label{ch:1}


Załóżmy, że Jan chce wysłać przez Internet wiadomość W do Anny. Z różnych powodów musi podzielić
ją na n pakietów. Każdemu pakietowi nadaje kolejne numery i wysyła przez sieć. Komputer Anny po otrzymaniu
przesłanych pakietów musi poskładać je w całą wiadomość, ponieważ mogą one przychodzić w losowej
kolejności. Państwa zadaniem jest zaprojektowanie i zaimplementowanie odpowiedniego rozwiązania radzącego
sobie z tym problemem. Należy wybrać i zaimplementować zgodnie z danym dla wybranej struktury
ADT oraz przeanalizować czas działania - złożoność obliczeniową proponowanego rozwiązania.

\section{Rozwiązanie}\label{ch:2}

Kod źródłowy znajduje się  \href{https://github.com/PartyKusZ/PAMSI}{\textbf{tutaj}}.

\subsection{Idea}

W celu rozwiązania wyżej postawionego zadania napisany został program, którego działanie polega na odczytaniu treści wiadomości z pliku ,,message.txt'',
podzieleniu jej na 10-cio znakowe pakiety, nadanie im kluczy definiujących położenie w treści wiadomości, zasymulowaniu wysłania pakietów, zasymulowaniu 
odebrania pakietów, posortowania i złożenia ich w jedną całą odebraną wiadomość zapisaną w pliku ,,rec\_message.txt''.

\subsection{Struktura danych}

\subsubsection{Kolejka priorytetowa}

W celu przechowywania pakietów w pamięci komputera została zastosowana kolejka priorytetowa. Zaimplementowana została struktura danych ,,t\_priority\_queue '' działająca na podstawie listy jednokierunkowej, dodatkowo z możliwością sortowania 
poprzez wstawianie elementów w odpowiednie miejsca, porównując ich klucze.
\subsubsection{Uzasadnienie wyboru}
 Wybór takiej struktury danych wynika ze specyfiki zadnia - wiadomości posiadające klucz identyfikacyjny mogą być dostarczane w losowej kolejności.
Ich liczba jest uzależniona od długości wiadomości (argument za zastosowaniem czegoś co działa na podstawie listy) i posiadają klucz (konieczność sortowania). 
Do rozwiązania tak postawionego problemu najlepiej nadaje się kolejka priorytetowa.  
Oto jej definicja:\\
\small{
\lstinputlisting[language=C++]{../inc/priority_queue.hpp}}

Metodami pozwalającymi wykonywać operacje na obiektach klasy ,,t\_priority\_queue'' są:
\begin{itemize}
    \item insert() - dodaje element do kolejki ustawiając jego położenie na podstawie klucza,
    \item pop() - usuwa element z początku kolejki,
    \item pop\_all() - usuwa całą kolejkę,
    \item empty() - orzeka, czy kolejka jest pusta,
    \item top() - zwraca wartość pierwszego elementu,
    \item size() - zwraca ilość elementów w kolejce, 
\end{itemize}

\subsection{Sortowanie}

Sortowanie danych odbywa się w niezwykle prosty sposób. Do kolejki wysyłana jest dana wraz z kluczem, metoda insert()
przesuwa wskaźnik po kolejnych elementach kolejki. Jeśli funkcja orzekająca określająca relacje pomiędzy kolejnymi kluczami kolejki a
podanym do insert() kluczem stwierdzi, iż klucz podany przestaje spełniać określoną relację (>,<), insert()
utworzy nowy węzeł i umieści nowy element we właściwym miejscu. Poniżej znajduje definicja insert():

\begin{lstlisting}
    void t_priority_queue<T> :: insert(const T &val, const int &x){

    if(data == nullptr){
        data = new str_of_data;
        data->T_type = val;
        quantity++;
        data->key = x;
    }else{
        str_of_data *tmp;
        str_of_data *tmpnew;
        str_of_data *tmpprev;
        

        tmp = data;
        while(this->comprasion(x,tmp->key) && tmp->next != nullptr){

            tmp = tmp->next;

        }

        if(this->comprasion(x,tmp->key)){
            tmp->next = new str_of_data;
            tmp->next->T_type = val;
            tmp->next->key = x;
            tmp->next->previous = tmp;
            quantity++;
        }else{
            if(tmp->previous == nullptr){
                tmpnew = new str_of_data;
                tmpnew->T_type = val;
                tmpnew->key = x;
                tmpnew->next = tmp;
                tmp->previous = tmpnew;
                data = tmp->previous;
            }else{
            tmpnew = new str_of_data;
            tmpnew->T_type = val;
            tmpnew->key = x;
            tmpnew->next = tmp;
            tmpnew->previous = tmp->previous;
            tmp->previous->next = tmpnew;
            tmp->previous = tmpnew;
            if(tmp == data)
                data = tmp->previous;
            quantity++;
            }
        }        
        
    }

}
\end{lstlisting}

\section{Złożoności obliczeniowe}

\begin{multicols}{2}

    \subsection{insert()}
    

\begin{figure}[H]
    \centering
    \begin{tikzpicture}[scale=1.5]
    \begin{axis}[
    title={Czas wykonywania metody insert() w funkcji ilości elementów},
    title style={text width=16em},
    xlabel={ilości elementów},
    ylabel={czas[ns]},
    xmin=0,xmax=10000,
    ymin=-1,ymax=4000,
    ymajorgrids=true,grid style=dashed
    ]

  
    \addplot[color=red] table[ignore chars={(,)},col sep=comma]{graph_insert_nanoseconds.tex};

  
    \end{axis}
    \end{tikzpicture}
\end{figure}
Metoda insert() w celu dodania elementu we właściwe miejsce, musi wykonać odpowiednio wiele przejść po elementach kolejki 
aby natrafić na element, którego klucz nie spełnia właściwej relacji (>,<). Jeśli element, przy którym relacja (>,<)
przestaje być spełniona znajduje się w odległości $n$ węzłów od początku kolejki, to kolejka musi wykonać $n$ przejść 
przez swoją listę. Oznacza to, iż ilość przejść jest wprost proporcjonalna do odległości ów węzła. Z tego wynika, iż
złożoność obliczeniowa jest określona funkcją liniową. Notacja dużego O:

{\Large \begin{equation*}
    O(n)
\end{equation*} }\\
\textbf{Uwaga}. W celu ukazania liniowej złożoności obliczeniowej na każdy kolejny element 
dodawany jest z wyższą wartością klucza od największego klucza w kolejce, aby każde kolejne dodanie elementu 
wiązało się z przebyciem drogi przez całą kolejkę.

\subsection{pop()}
\vspace{1.51cm}

\begin{figure}[H]
    \centering
    \begin{tikzpicture}[scale=0.7]
    \begin{axis}[
    title={Czas wykonywania metody pop() w funkcji ilości elementów},
    title style={text width=16em},
    xlabel={ilości elementów},
    ylabel={czas[ns]},
    xmin=0,xmax=10000,
    ymin=-2,ymax=100,
    legend pos=north west,
    ymajorgrids=true,grid style=dashed
    ]
    
    \addplot[color=red] table[ignore chars={(,)},col sep=comma]{graph_pop_nanoseconds.tex};
    
   
    \end{axis}
    \end{tikzpicture}
    
    

\end{figure}

Metoda pop() w celu zdjęcia pierwszego elementu musi wykonać zawsze 
tylko jedną operację, więc jej złożoność jest stała. Notacja dużego O:

{\Large \begin{equation*}
    O(1)
\end{equation*} }
\newpage
\subsection{pop\_all()}

\begin{figure}[H]
    \centering
    \begin{tikzpicture}[scale=1.5]
    \begin{axis}[
    title={Czas wykonywania metody pop\_all() w funkcji ilości elementów},
    title style={text width=16em},
    xlabel={ilości elementów},
    ylabel={czas[ns]},
    xmin=0,xmax=10000,
    ymin=-1,ymax=50,
    ymajorgrids=true,grid style=dashed
    ]

  
    \addplot[color=red] table[ignore chars={(,)},col sep=comma]{graph_pop_all_microseconds.tex};

  
    \end{axis}
    \end{tikzpicture}
\end{figure}

Metoda pop\_all() w celu usunięcia wszystkich elementów musi wykonać tyle operacji ile jest węzłów w kolejce,
z czego wynika że jej złożoność obliczeniowa jest liniowa. Notacja dużego O:

{\Large \begin{equation*}
    O(n)
\end{equation*} }


\subsection{empty()}

\begin{figure}[H]
    \centering
    \begin{tikzpicture}[scale=1.5]
    \begin{axis}[
    title={Czas wykonywania metody empty() w funkcji ilości elementów},
    title style={text width=16em},
    xlabel={ilości elementów},
    ylabel={czas[ns]},
    xmin=0,xmax=2000,
    ymin=-2,ymax=500,
    ymajorgrids=true,grid style=dashed
    ]

  
    \addplot[color=red] table[ignore chars={(,)},col sep=comma]{graph_empty_nanoseconds.tex};

  
    \end{axis}
    \end{tikzpicture}
\end{figure}

Metoda empty() w celu zwrócenia informacji o tym czy kolejka jest pusta musi wykonać zawsze 
tylko jedną operację, więc jej złożoność jest stała. Notacja dużego O:

{\Large \begin{equation*}
    O(1)
\end{equation*} }

\subsection{top()}

\begin{figure}[H]
    \centering
    \begin{tikzpicture}[scale=1.5]
    \begin{axis}[
    title={Czas wykonywania metody top() w funkcji ilości elementów},
    title style={text width=16em},
    xlabel={ilość elementów },
    ylabel={czas[ns]},
    xmin=0,xmax=100,
    ymin=0,ymax=120,
    legend pos=north west,
    ymajorgrids=true,grid style=dashed
    ]
    
    \addplot[color=red] table[ignore chars={(,)},col sep=comma]{graph_top_nanoseconds.tex};
    

    \end{axis}
    \end{tikzpicture}
    
    

\end{figure}

Metoda top() w celu zwrócenia wartości elementu znajdującego się na początku kolejki zawsze musi wykonać tylko jedną operacje,
wiec złożoność obliczeniowa jest stała. Notacja dużego O:

{\Large \begin{equation*}
    O(1)
\end{equation*} }

\subsection{size()}
\begin{figure}[H]
    \centering
    \begin{tikzpicture}[scale=0.7]
    \begin{axis}[
    title={Czas wykonywania metody size() w funkcji ilości elementów},
    title style={text width=16em},
    xlabel={ilości elementów},
    ylabel={czas[ns]},
    xmin=0,xmax=2000,
    ymin=-1,ymax=1000,
    ymajorgrids=true,grid style=dashed
    ]

  
    \addplot[color=red] table[ignore chars={(,)},col sep=comma]{graph_size_nanoseconds.tex};

  
    \end{axis}
    \end{tikzpicture}
\end{figure}

Metoda size() w celu zwrócenia wartości ilości elementów w kolejce zawsze musi wykonać tylko jedną operacje,
wiec złożoność obliczeniowa jest stała. Notacja dużego O:

{\Large \begin{equation*}
    O(1)
\end{equation*} }

    
\end{multicols}
\section{Podsumowanie i wnioski}

Poprzez zaimplementowanie kolejki priorytetowej udało się rozwiązać zadanie postawione w pkt.(\ref{ch:1}), co 
prezentuje driver znajdujący się w podanym w pkt.(\ref{ch:2}) repozytorium. Wybór kolejki priorytetowej uważam za jak najbardziej 
uzasadniony, specyfika jej działania idealnie sprawdza się w przypadkach, gdy należy sortować ciągle napływające dane, nie znając do tego ich ilości.\\

Szybkość działania kolejki jest na bardzo wysokim poziome, powyższa implementacja gwarantuje liniową zależność 
czasową podczas dodawania elementu jak i podczas usuwania całej kolejki, pozostałe metody są stałe w czasie. 



\end{document}
