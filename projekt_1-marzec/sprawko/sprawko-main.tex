\documentclass[12pt]{article}
\usepackage[utf8]{inputenc}
\usepackage[T1]{fontenc}
\usepackage[polish]{babel}
\usepackage{geometry}
\usepackage{tabularx}
\usepackage[table,xcdraw]{xcolor}
\usepackage{color}
\usepackage{subfig}
\usepackage{sidecap}
\usepackage{wrapfig}
\usepackage{float}
\usepackage{enumerate}
\usepackage{graphicx}
\usepackage{multirow}
\setlength{\parindent}{0pt}
\usepackage{hyperref}
\usepackage{titlesec}
\titlelabel{\thetitle.\quad}
\usepackage{amsmath}
\usepackage{anyfontsize}
\usepackage{indentfirst}
\usepackage{listings}
\usepackage{multicol}
\usepackage{pgfplots}
%\pgfplotsset{compat=1.16}
\pgfplotsset{ticks=none}
\newgeometry{tmargin=1.8cm,bmargin=1.8cm,lmargin =1.8cm,rmargin=1.8cm}

\begin{document}
    
\begin{titlepage}
\begin{figure}
	\centering
	\includegraphics[width=18cm]{//home/kubus/Obrazy/logo-Pwr.png}
	
	\label{fig:pwr}
\end{figure}
	\begin{center}
		\huge Wydział Elektroniki, Fotoniki i Mikrosystemów \\ 
		\vspace{40pt}
		\huge PAMSI  \\
	\end{center}
	\vspace{60pt}
	\hrule
	\vspace{1pt}
	\hrule
	\begin{center}
		{\fontsize{40}{50}\selectfont Sprawozdanie nr 3\\ }
		\vspace{10pt}
		{\fontsize{25}{25}\selectfont Implementacja algorytmu MiniMax z alfa-beta cięciami w grze kółko i krzyżyk }
	\end{center}
	\hrule
	\vspace{1pt}
	\hrule
	\begin{flushright}
		\vspace{65pt}
		\textit{\Large Prowadzący:}\\
		
		\Large dr hab. inż. Andrzej Rusiecki\\
		\vspace{10pt}
		\textit{\Large Wykonał:}\\
		
		\Large Jakub Kusz \\
	
	\end{flushright}
	\vspace{90pt}
	\begin{center}
		\large Wrocław, \today r.
	\end{center}
\end{titlepage}



\section{Zadanie}


Załóżmy, że Jan chce wysłać przez Internet wiadomość W do Anny. Z różnych powodów musi podzielić
ją na n pakietów. Każdemu pakietowi nadaje kolejne numery i wysyła przez sieć. Komputer Anny po otrzymaniu
przesłanych pakietów musi poskładać je w całą wiadomość, ponieważ mogą one przychodzić w losowej
kolejności. Państwa zadaniem jest zaprojektowanie i zaimplementowanie odpowiedniego rozwiązania radzącego
sobie z tym problemem. Należy wybrać i zaimplementować zgodnie z danym dla wybranej struktury
ADT oraz przeanalizować czas działania - złożoność obliczeniową proponowanego rozwiązania.

\section{Rozwiązanie}

Kod źródłowy znajduje się  \href{https://github.com/PartyKusZ/PAMSI}{\textbf{tutaj}}.

\subsection{Idea}

W celu rozwiązania wyżej postawionego zadania napisany został program, którego działanie polega na odczytaniu treści wiadomości z pliku ,,message.txt'',
podzieleniu jej na 10-cio znakowe pakiety, nadanie im kluczy definiujących położenie w treści wiadomości, zasymulowaniu wysłania pakietów, zasymulowaniu 
odebrania pakietów, posortowania i złożenia ich w jedną całą odebraną wiadomość zapisaną w pliku ,,rec\_message.txt''.

\subsection{Struktura danych}

W celu przechowywania pakietów w pamięci komputera została napisana struktura danych \\,,t\_priority\_queue '' działająca na podstawie listy jednokierunkowej, dodatkowo z możliwością sortowania,
na kształt kolejki priorytetowej. Wybór takiej struktury danych wynika ze specyfiki zadnia - wiadomości posiadające klucz identyfikacyjny mogą być dostarczane w losowej kolejności.
Ich liczba jest uzależniona od długości wiadomości (argument za zastosowaniem czegoś co działa na kolejce) i posiadają klucz (konieczność sortowania). 
Do rozwiązania tak postawionego problemu najlepiej nadaje się kolejka priorytetowa.  
Oto jej definicja:\\
\small{
\lstinputlisting[language=C++]{../inc/priority_queue.hpp}}

Metodami pozwalającymi wykonywać operacje na obiektach klasy ,,t\_priority\_queue'' są:
\begin{itemize}
    \item push() - dodaje element na koniec kolejki,
    \item pushs() - dodaje element na koniec kolejki i sortuje kolejkę wg kluczy,
    \item pop() - usuwa element z początku kolejki,
    \item pop\_all() - usuwa całą kolejkę,
    \item empty() - orzeka, czy kolejka jest pusta,
    \item top() - zwraca wartość pierwszego elementu,
    \item size() - zwraca ilość elementów w kolejce,
    \item sort() - sortuje kolejkę wg kluczy. 
\end{itemize}

\subsection{Sortowanie}

Jako algorytm sortujący kolejkę został wybrany \textbf{quick sort}.

\subsubsection{Zasada działania}

Algorytm wykorzystuje technikę "dziel i zwyciężaj". Według ustalonego schematu wybierany jest jeden element w sortowanej 
tablicy, który będziemy nazywać pivot. Pivot może być elementem środkowym, pierwszym, ostatnim, 
losowym lub wybranym według jakiegoś innego schematu dostosowanego do zbioru danych. 
Następnie ustawiamy elementy nie większe na lewo tej wartości, 
natomiast nie mniejsze na prawo. 
W ten sposób powstaną nam dwie części tablicy (niekoniecznie równe), gdzie w 
pierwszej części znajdują się elementy nie większe od drugiej. Następnie każdą z tych 
podtablic sortujemy osobno według tego samego schematu. 

\subsubsection{Implementacja}

W programie algorytm qiuck sort został zaimplementowany w sposób rekurencyjny:
\begin{lstlisting}[language = C++]
  
    template<typename T>

void t_vector<T> :: sort(const int &left,const int &right){

    template<typename T>

    void t_vector<T> :: sort(){
    
        str_of_data tmp;
    
        int i = 0;
        int j = quantity - 1;
        int pivot = quantity / 2;
        do{
            while(comprasion_ascending((*this)[pivot],(*this)[i])){
                i++;
            }
            while(comprasion_ascending((*this)[j],(*this)[pivot])){
                j--;
            }
            if(i <= j){
                tmp = (*this)[i];
                (*this)[i] = (*this)[j];
                (*this)[j] = tmp;
                
                i++;
                j--;
            }
        }while(i <= j);
         if(0 < j)
           this->sort(0,j);
        if(quantity - 1 > i)
            this->sort(i,quantity - 1);
    
    };
\end{lstlisting}

\begin{lstlisting}
    

    bool comprasion_ascending(str_of_data x, str_of_data y){return x.key > y.key;};

\end{lstlisting}

\textbf{Uwaga.} W kodzie znajduje się również druga wersja metody sort(), przyjmująca 
argumenty określające zakres sortowania. Wersja bez argumentu domyślnie sortuje całą kolejkę, następnie rekurencyjnie wywołuje 
wersję z argumentami zakresu.


\section{Złożoności obliczeniowe}

W celu określenia złożoności obliczeniowej poszczególnych operacji na kolejce, zostały przeprowadzone testy
szybkości w zależności od ilości elementów znajdujących się w kolejce. Testom poddano każdą metodę 
w zakresie od 1 do 10000 elementów w kolejce i sporządzono wykresy.

\begin{multicols}{2}
    \subsection{push()}


\begin{figure}[H]
    \centering
    \begin{tikzpicture}[scale=0.7]
    \begin{axis}[
    title={Czas wykonywania metody push() w zależności od  ilości elementów},
    title style={text width=16em},
    grid=both,
    xlabel={ilość elementów},
    ylabel={czas[us]},
    xmin=0,xmax=10000,
    ymin=0,ymax=50000,
    legend pos=north west,
    ymajorgrids=true,grid style=dashed
    ]
    
    \addplot[color=red] table[ignore chars={(,)},col sep=comma]{graph_push_nanoseconds.tex};
    
    
    \end{axis}
    \end{tikzpicture}
    
    

    \label{graph: push}
\end{figure}
{\small
Poszczególne czasy wykonywania metody push() różnią się od siebie czasem wykonania, lecz \textbf{złożoność czasowa jest stała}.

}
\Large{
\begin{equation*}
    O(1)
\end{equation*}}

\subsection{pushs()}

\begin{figure}[H]
    \centering
    \input{wykres_pushs.tex}
    \label{graph: pushs}    
\end{figure}
{\small Poszczególne czasy wykonywania metody pushs() różnią się od siebie czasem wykonania. Z powyższego wykresu wynika, iż \textbf{złożoność czasowa jest kwadratowa}.}
{\Large
\begin{equation*}
    O(n^2)
\end{equation*}}

\subsection{pop()}
\begin{figure}[H]
    \centering
   \begin{tikzpicture}[scale=0.7]
    \begin{axis}[
    title={Czas wykonywania metody pop() w funkcji ilości elementów},
    title style={text width=16em},
    xlabel={ilości elementów},
    ylabel={czas[ns]},
    xmin=0,xmax=10000,
    ymin=-2,ymax=100,
    legend pos=north west,
    ymajorgrids=true,grid style=dashed
    ]
    
    \addplot[color=red] table[ignore chars={(,)},col sep=comma]{graph_pop_nanoseconds.tex};
    
   
    \end{axis}
    \end{tikzpicture}
    
    

    \label{graph: pop}
    
\end{figure}
Poszczególne czasy wykonywania metody pop() różnią się od siebie czasem wykonania, lecz \textbf{złożoność czasowa jest stała}.
\Large{
\begin{equation*}
    O(1)
\end{equation*}}


\subsection{pop\_all()}
\begin{figure}[H]
    \centering
    \begin{tikzpicture}[scale=0.7]
    \begin{axis}[
    title={Czas wykonywania metody pop\_all() w funkcji ilości elementów},
    title style={text width=16em},
    xlabel={ilość elementów},
    ylabel={Yczas[us]},
    xmin=0,xmax=10000,
    ymin=-1,ymax=500,
    legend pos=north west,
    ymajorgrids=true,grid style=dashed
    ]
    
  \addplot[color=red] table[ignore chars={(,)},col sep=comma]{graph_pop_all_microseconds.tex};
    \end{axis}
    \end{tikzpicture}
    
    

    \label{graph: pop_all}
    
\end{figure}
Poszczególne czasy wykonywania metody pop\_all() różnią się od siebie czasem wykonania. Z powyższego wykresu wynika, iż \textbf{złożoność czasowa jest liniowa}.
\Large{
\begin{equation*}
    O(n)
\end{equation*}}

\subsection{empty()}

\begin{figure}[H]
    \centering
   \begin{tikzpicture}[scale=1.5]
    \begin{axis}[
    title={Czas wykonywania metody empty() w funkcji ilości elementów},
    title style={text width=16em},
    xlabel={ilości elementów},
    ylabel={czas[ns]},
    xmin=0,xmax=2000,
    ymin=-2,ymax=500,
    ymajorgrids=true,grid style=dashed
    ]

  
    \addplot[color=red] table[ignore chars={(,)},col sep=comma]{graph_empty_nanoseconds.tex};

  
    \end{axis}
    \end{tikzpicture}
    \label{graph: empty}
    
\end{figure}
Poszczególne czasy wykonywania metody empty() różnią się od siebie czasem wykonania, lecz \textbf{złożoność czasowa jest stała}.
\Large{
\begin{equation*}
    O(1)
\end{equation*}}

\subsection{top()}

\begin{figure}[H]
    \centering
    \begin{tikzpicture}[scale=1.5]
    \begin{axis}[
    title={Czas wykonywania metody top() w funkcji ilości elementów},
    title style={text width=16em},
    xlabel={ilość elementów },
    ylabel={czas[ns]},
    xmin=0,xmax=100,
    ymin=0,ymax=120,
    legend pos=north west,
    ymajorgrids=true,grid style=dashed
    ]
    
    \addplot[color=red] table[ignore chars={(,)},col sep=comma]{graph_top_nanoseconds.tex};
    

    \end{axis}
    \end{tikzpicture}
    
    

    \label{graph: top}
    
\end{figure}
Poszczególne czasy wykonywania metody top() różnią się od siebie czasem wykonania, lecz \textbf{złożoność czasowa jest stała}.
\Large{
\begin{equation*}
    O(1)
\end{equation*}}

\subsection{size()}

\begin{figure}[H]
    \centering
    \begin{tikzpicture}[scale=0.7]
    \begin{axis}[
    title={Czas wykonywania metody size() w funkcji ilości elementów},
    title style={text width=16em},
    xlabel={ilości elementów},
    ylabel={czas[ns]},
    xmin=0,xmax=2000,
    ymin=-1,ymax=1000,
    ymajorgrids=true,grid style=dashed
    ]

  
    \addplot[color=red] table[ignore chars={(,)},col sep=comma]{graph_size_nanoseconds.tex};

  
    \end{axis}
    \end{tikzpicture}
    \label{graph: size}
    
\end{figure}
Poszczególne czasy wykonywania metody size() różnią się od siebie czasem wykonania, lecz \textbf{złożoność czasowa jest stała}.
\Large{
\begin{equation*}
    O(1)
\end{equation*}}

\subsection{sort()}

\begin{figure}[H]
    \centering
    \begin{tikzpicture}[scale=1.5]
    \begin{axis}[
    title={Czas wykonywania metody sort() w funkcji ilości elementów},
    title style={text width=16em},
    xlabel={ilośc elementów },
    ylabel={czas},
    xmin=0,xmax=100,
    ymin=0,ymax=120,
    legend pos=north west,
    ymajorgrids=true,grid style=dashed
    ]
    
    \addplot[color=red] table[ignore chars={(,)},col sep=comma]{graph_sort_minutes.tex};
 
    \end{axis}
    \end{tikzpicture}
    
    

    \label{graph: sort}
    
\end{figure}
{\small Poszczególne czasy wykonywania metody sort() różnią się od siebie czasem wykonania. Z powyższego wykresu wynika, iż \textbf{złożoność czasowa jest kwadratowa}.}
\Large{
\begin{equation*}
    O(n^2)
\end{equation*}}
\end{multicols}

\section{Wnioski}



\end{document}
